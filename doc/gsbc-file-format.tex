\documentclass{article}
\title{Global Script Byte-Code File Format}
\author{Jonathan Cast\\\texttt{<jonathanccast@fastmail.fm>}}

\begin{document}

\maketitle

\section{Overview}

\begin{itemize}
  \item Header
  \item Code Segment
  \item Data Segment
  \item Strings
\end{itemize}
NB: 2011-02-22: Ordering of sections differs between spec and loader; will need to fix loader.  End NB

\section{Header}

\begin{tabular}{cl}
  & Optional \verb+#!+ line (max 80 bytes (not chars)). \\
  & Following offsets are relative to end of this line, if present.  \\
\texttt{0x00} & Magic Number \verb+/!gs(pre|doc)bc/+ (8 bytes) \\
\texttt{0x08} & File Format Version Number (4 bytes; Era.Major.Minor.Step) \\
\texttt{0x0C} & Size of Headers (exclusive of \verb+#!+ line) (4 bytes) \\
\texttt{0x10} & Size of Code (4 bytes) \\
\texttt{0x14} & Size of Data (4 bytes) \\
\texttt{0x10} & Size of Strings (4 bytes) \\
\end{tabular}

(Total size of header (inclusive of \verb+#!+ line) will be between 0x1C and 0x6C lines).

\section {Byte Codes}

\begin{tabular}{rl@{ --- }l}
  00 & \texttt{ALLOC} & Append to data segment \\
  01 & \texttt{ANALYZE} & Push an \texttt{analyze} frame \\
  02 & \texttt{APP} & Push an argument frame \\
  03 & \texttt{CALL} & Call a known function \\
  04 & \texttt{CCALL} & Call a manifest closure \\
  05 & \texttt{EVAL} & Enter a thunk \\
  06 & \texttt{EEXEC} & Call a subprogram expression \\
  07 & \texttt{EXEC} & Call a manifest subprogram \\
  08 & \texttt{RETURN} & Return from a manifest block \\
  09 & \texttt{PRIM} & Apply a primitive \\
  0A & \texttt{TYALLOC} & Manufacture a type at runtime \\
  & \multicolumn{2}{c}{Declarations} \\
  & \texttt{.constr} & The code label for a \texttt{\%constr} \\
  & \texttt{.arg} & Declare a formal parameter \\
  & \texttt{.free} & Declare a free variable \\
  & \texttt{.lambda} & Declare a manifest function \\
  & \texttt{.lprog} & Declare a manifest sub-routine \\
  & \texttt{.prog} & Declare a manifest subprogram \\
  & \texttt{.proge} & Declares a block expression \\
  & \texttt{.lproge} & Delcares a block expression-bodied lambda \\
  & \texttt{.tyarg} & Declares a type parameter \\
  & \texttt{.tyfree} & Declares a free type variable \\
\end{tabular}

The final operation in a block must not take an address; the initial declaration must take an address.
\texttt{APP} and \texttt{ANALYZE} may not take an address.
Otherwise, addresses are always optional.
Addresses must be unique across the entire source file;
only addresses of blocks annotated \texttt{.public} must be unique across all files.

%% TODO
%%
%% * Convert to Global Script

\end{document}
