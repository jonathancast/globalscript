\documentclass{article}

\usepackage{haskell}

\title{Memory Design -- IBIO Implementation}
\author{Jonathan Cast\\\texttt{jcast@globalscript.org}}

\newcommand\textc[1]{\texttt{#1}}
\newcommand\cfile[1]{\texttt{#1}}
\newcommand\gssumtype[1]{.\Sigma\langle\hsbody{#1}\rangle}
\newcommand\gsemptyuproduct{.u\Pi\langle\rangle}
\newcommand\gsemptyproduct{.\Pi\langle\rangle}

\begin{document}

\maketitle

\section{Procedure for Reading From a Unix File in IBIO}

This is wrong;
we need to use combinatorial acceptors to be able to distinguish
between \<iptr\>s held by the input mechanism and \<iptr\>s held by Global Script.

\begin{itemize}
    \item The read thread drives
    
    Read threads have these operations:
    \begin{haskell}
        tag :: \forall '\alpha. iptr \alpha \to m \gssumtype{
            symbol \gsemptyuproduct; \\
            unix.eof \gsemptyuproduct; \\
            ibio.eof \gsemptyuproduct;
        }; \\
        get :: \forall '\alpha. iptr \alpha \to m \alpha; \\
        advance :: \forall '\alpha. iptr \alpha \to m (iptr \alpha); \\
        accept :: \forall '\alpha. iptr \alpha \to m \gsemptyproduct;
    \end{haskell}
\end{itemize}

\<tag\>, \<get\>, and \<advance\> all demand their argument; \<accept\> does not.
Under the hood, an \<iptr.t\> is a pointer into the IBIO buffer.
To demand an \<iptr.t\>, you follow this procedure:
\begin{itemize}
    \item Lock the buffer.
    \item Check the \<iptr.t\>'s type.
    
    Four types:
    \begin{itemize}
        \item Points at a symbol.
        
        Nothing to do; un-lock the buffer.
        
        \item Points at a Unix EOF break.
        
        Nothing to do; un-lock the buffer.
        
        \item Points at an internal EOC.
        
        Huh?  Anyway, nothing to do; un-lock the buffer.
        
        \item Points at EOD, but not at an internal EOC.
        
        Continue.
    \end{itemize}

    \item Lock the UxIO buffer.
    
    \item Check for adequate input.
    
    If present, copy a single symbol into the IBIO buffer and
\end{itemize}

\section{Memory Layout}

\subsection{Input from a Device}

\begin{tabular}{ll}
    UxIO & \begin{tabular}{|c|c|c|c|}
        \hline
        sched & perf & blank & bef \\
        \hline
    \end{tabular} \\
    \\
    IBIO & \begin{tabular}{|c|c|c|}
        \hline
        sched & perf & free \\
        \hline
    \end{tabular}
\end{tabular}

Scheduled input in the UxIO buffer can be ignored.
Basically, it's free.
Performed but not scheduled input is held in the UxIO buffer until scheduled.
Thus:

\begin{tabular}{ll}
    UxIO & \begin{tabular}{|c|c|c|}
        \hline
        free & perf & free \\
        \hline
    \end{tabular}
\end{tabular}

So the UxIO channel needs
\begin{itemize}
    \item File descriptor
    \item Beginning of pre-fetched input
    \item End of pre-fected input
\end{itemize}

Or:

\begin{tabular}{ll}
    UxIO & \begin{tabular}{|c|c|c|}
        \hline
        perf & free & perf \\
        \hline
    \end{tabular}
\end{tabular}

Which can be detected because the `end' of the performed input is before the `beginning'.

So a full channel consists of:
\begin{itemize}
    \item An IBIO channel descriptor, which contains channel metadata and a pointer to the end of accepted input and either the end of scheduled input (for an external channel) or the end of scheduled output (for an internal channel).
    \item An IBIO buffer, which is a linked list of 1kw memory regions.
    \item (For external channels) a UxIO channel descriptor, which contains a file descriptor, pointers to the beginning and end of performed input, and possibly other data.
    \item A UxIO buffer, which is a 64kB circular buffer used for reads.
\end{itemize}

\end{document}
