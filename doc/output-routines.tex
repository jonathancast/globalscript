\documentclass{article}
\title{Output Routines}
\author{Jonathan Cast\\\texttt{<jcast@globalscript.org>}}

\usepackage{haskell}

\newcommand\instance{\hskwd{instance}}

\begin{document}

\maketitle

Note: this is just a quick list of routines; I'll add rationale later.  End note.

Note: not all the output routines listed here exist at any types yet.  End note.

Note: many of the output routines listed here are overloaded;
in general, the instance of \<f\> at a type \<m.t\> will be called \<m.f\> in hand-written \texttt{.ags} and \texttt{.cgs} files.

Note: all the output routines listed here return \<list.t rune.t\>, which is probably the wrong type;
but I expect it's fine for now.
TODO: Change all the output routines to take \<difference.list.t rune.t\>,
and change \<ibio.send\> to take \<difference.list.t o\>.
End note.

\section{Formatted Output}

\subsection{\<fmtdecimal\>, \<fmtoctal\>, and \<fmthex\>}

\begin{haskell}
    fmtdecimal, fmtoctal, fmthex, :: '\alpha{} \to{} list.t rune.t;
\end{haskell}
$\alpha$ is e.g. \<natural.t\>, \<rational.t\>, etc.
(\<real.t\> cannot be formatted without both a precedence and a non-determinism monad.)

\subsection{\<fmtquote\>, \<fmtquotes\>}
\begin{haskell}
    fmtquote &:: list.t rune.t \to{} list.t rune.t; \\
    fmtquotes &:: list.t (list.t rune.t) \to{} list.t rune.t;
\end{haskell}
These take a string or list of strings and returns the rc-quoted version.

\<fmtquotes\> is:
\begin{haskell}
    'fmtquotes 'ts = list.concat \$ list.intersperse qq\{ \} \$ map fmtquote ts;
\end{haskell}

Note: with these types, you need to be very careful to segregate quoted and unquoted variables manually.  End note.

\section{Macro-scale Output}

\begin{haskell}
    list.<> :: list.t rune.t \to{} list.t rune.t \to{} list.t rune.t;
\end{haskell}
It may seem obvious, but of course this routine, as well as string literals and interpolation, plays a large role in IBIO output.

\begin{haskell}
    fmtline &:: list.t rune.t \to{} list.t rune.t; \\
    fmtlines &:: list.t (list.t rune.t) \to{} list.t rune.t;
\end{haskell}
\<fmtline\> ensures that the string given ends in a newline; \<fmtlines\> does the same for a list of strings.

\section{Direct Output}

\begin{haskell}
    print :: '\alpha{} \to{} list.t rune.t;
\end{haskell}
This formats an \<\alpha\> the way it would normally appear in a file.
So, e.g.,
\begin{haskell}
    \instance{} print rune.t 'r &= r : nil; \\
    \instance{} print (list.t rune.t) 's &= s; \\
    \instance{} print file.name.t 'fn &= fmtquote \$ file.name.out fn;
\end{haskell}

\section{Debugging Output}

\begin{haskell}
    fmtgs :: '\alpha{} \to{} list.t rune.t;
\end{haskell}
This returns (a) source syntax for the value, suitable for sending back to a Global Script REPL.
This is the closest equivalent to Haskell's \<show\> function.

\end{document}
