\documentclass{article}

\title{Memory Design -- ACE/API Global Script Implementation}
\author{Jonathan Cast\\\texttt{jcast@globalscript.org}}

\newcommand\textc[1]{\texttt{#1}}
\newcommand\cfile[1]{\texttt{#1}}

\begin{document}

\maketitle

The ACE memory manager (the Simple Segment Manager) controls some portion of the data segment starting at an address above that used for static C components.
The SSM lives in \cfile{libglobalscript/alloc.c}.
It divides this memory into 256kw segments, each of which is assigned to a particular client.

Terminology:
\begin{itemize}
    \item Segment --- a 256kw block of memory, managed by the ASM and provided to a particular ASM client.
    \item Class --- a homogenous class of segments, employing the same set of routines for evaluation and garbage collection.
    \item Sub-class --- a sub-set of the segments in a given class, assigned an arbitrary meaning by 
\end{itemize}

The SSM assigns segments to clients and maintains the evaluation and garbage collection data.
The clients are responsible for managing the segments they own and returning them to the SSM when appropriate.
The SSM itself acts as client for the free segments.

% FIXME Move this to the proper document
Evaluation looks like this:
\begin{itemize}
    \item The status function, which requests evaluation if necessary (fast) and reports the evaluation status of a cell.

    The status function can return these results:
    \begin{itemize}
        \item Un-evaluated (cannot happen)
        \item Evaluating
        \item Blocked (return a universal scheduler blocking token)
        \item Evaluated
        \item Indirection (return a \textc{gsvalue} containing the location of the actual value)
    \end{itemize}
\end{itemize}

Metadata associated to a segment:

\begin{itemize}
    \item Evaluation function
    \item Trace function (for garbage collection)
\end{itemize}

Each segment begins with the metadata, so clients need to account for that!

Free segments look like this (word offsets)

\begin{itemize}
    \item[0x00] Evalulation function is \textc{gsfree}.
    \item[0x01] Trace function is \textc{gstracefree}.
    \item[0x02] Pointer to the next free segment.
\end{itemize}

We keep a pointer to the least free segment.
This pointer is almost always non-null; it is an error for \textc{gs\_sys\_seg\_alloc} to be called when it is null.

The SSM has three entry points:
\begin{itemize}
    \item \textc{gs\_sys\_seg\_init} --- Called from \textc{main}; do not call
    \item \textc{gs\_sys\_seg\_alloc} --- Retrieves the least segment from the free list, initializes it, and returns it.
        Keep the arguments to this function synchronized with \textc{struct gs\_blockdesc}.
        If this function would empty the free list, it attempts to expand the data segment; if that fails, it schedules an emergency garbage collection.
        It is an error to call \textc{gs\_sys\_seg\_alloc} again before the garbage collection occurs.
    \item \textc{gs\_sys\_seg\_free} --- Returns a segment to the free list.
\end{itemize}

\end{document}
