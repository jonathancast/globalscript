\documentclass{article}
\title{API}
\author{Jonathan Cast\\\texttt{jcast@gobalscript.org}}

\newcommand\defn[1]{\emph{#1}}
\newcommand\TODO[1]{\footnote{TODO: #1}}

\begin{document}

\maketitle

``API'' stands for \defn{Advanced Procedural language Implementation}.
This is how IBIO, Cord, maybe others implement procedural language-style monads (similar to Haskell's IO monad).
Execution of API-based languages is divided into two phases:
\begin{itemize}
    \item \defn{Thread advancement} --- this means 
    \item
\end{itemize}

We have five kinds of OS threads/processes:
\begin{itemize}
    \item The \defn{Unix process set} consists of one process per `hard' API thread.
        A \defn{hard} API thread is not a thread that runs in its own process,
        but rather one that has a process to which any system calls it makes are dispatched, usually.
        The main thread starts out hard; other threads start out soft and are hardened `lazily'.
        The main thread
    \item The \defn{API thread pool} drives everything else that happens, including in the main processes.
    \item The \defn{ACE thread pool} does Global Script evaluation;
        it is started by the API thread pool and shut down when:
        \begin{itemize}
            \item The last API thread has exited
            \item The last read thead/process has exited\TODO{IBIO read processes/read threads}
            \item The last write thread/process has exited\TODO{IBIO write processes/write threads}
        \end{itemize}
\end{itemize}

\end{document}